%%%%%%%%%%%%%%%%%%%%%%%%%%%%%%%%%%%%%%%%%
% "ModernCV" CV and Cover Letter
% LaTeX Template
% Version 1.3 (29/10/16)
%
% This template has been downloaded from:
% http://www.LaTeXTemplates.com
%
% Original author:
% Xavier Danaux (xdanaux@gmail.com) with modifications by:
% Vel (vel@latextemplates.com)
%
% License:
% CC BY-NC-SA 3.0 (http://creativecommons.org/licenses/by-nc-sa/3.0/)
%
% Important note:
% This template requires the moderncv.cls and .sty files to be in the same 
% directory as this .tex file. These files provide the resume style and themes 
% used for structuring the document.
%
%%%%%%%%%%%%%%%%%%%%%%%%%%%%%%%%%%%%%%%%%

%----------------------------------------------------------------------------------------
%	PACKAGES AND OTHER DOCUMENT CONFIGURATIONS
%----------------------------------------------------------------------------------------

\documentclass[11pt,a4paper,sans]{moderncv} % Font sizes: 10, 11, or 12; paper sizes: a4paper, letterpaper, a5paper, legalpaper, executivepaper or landscape; font families: sans or roman

\usepackage[utf8]{inputenc}
\moderncvstyle{casual} % CV theme - options include: 'casual' (default), 'classic', 'oldstyle' and 'banking'
\moderncvcolor{blue} % CV color - options include: 'blue' (default), 'orange', 'green', 'red', 'purple', 'grey' and 'black'

\usepackage{lipsum} % Used for inserting dummy 'Lorem ipsum' text into the template
%\usepackage{fontspec}

\usepackage[scale=0.75]{geometry} % Reduce document margins
\setlength{\hintscolumnwidth}{4cm} % Uncomment to change the width of the dates column
%\setlength{\makecvtitlenamewidth}{10cm} % For the 'classic' style, uncomment to adjust the width of the space allocated to your name

%----------------------------------------------------------------------------------------
%	NAME AND CONTACT INFORMATION SECTION
%----------------------------------------------------------------------------------------

\firstname{Pablo} % Your first name
\familyname{Riutort Grande} % Your last name


%----------------------------------------------------------------------------------------

\begin{document}



%----------------------------------------------------------------------------------------
%	CURRICULUM VITAE
%----------------------------------------------------------------------------------------

\makecvtitle % Print the CV title

%----------------------------------------------------------------------------------------
%	INFO SECTION
%----------------------------------------------------------------------------------------

\section{Contacto}

\cvitem{Teléfono}{+34 693 948 133}
\cvitem{Email}{pablo.grande@proton.me}
\cvitem{Web}{\href{https://github.com/pablo-grande}{github.com/pablo-grande}}


%----------------------------------------------------------------------------------------
%	EDUCATION SECTION
%----------------------------------------------------------------------------------------

\section{Educación}

\cventry{Cursando}{Máster Universitario}{Máster interuniversitario en Ingeniería Computacional y Matemática}{URV-UOC}{}{}
\cventry{}{Máster Universitario}{Seguridad de las Tecnologías de la Información y de las Comunicaciones}{Interuniversitario: UOC, UAB, URV}{}{}
\cventry{}{Grado en Ingeniería Informática}{Mención en Tecnologías de la Información}{Universitat de les Illes Balears}{UIB}{}
\cventry{}{Beca EUROWEB+}{Computer Science and Informatics}{Faculty of Electronic Engineering}{Univerity of Nis}{} % Arguments not required can be left empty


%----------------------------------------------------------------------------------------
%	WORK EXPERIENCE SECTION
%----------------------------------------------------------------------------------------

\section{Experiencia}

\cventry{2021}{Specialist SW Network as a Platform}{\textsc{Vodafone}}{Madrid}{}{
\begin{itemize}
\item Desarrollador backend en Flask y Kubernetes.
\item Desarrollo de proyectos de innovación con MQTT, Python y JavaScript.
\item Mantenimiento de proyectos y core con Python y bash.
\end{itemize}
}
\cventry{2019 - 2021}{Desarrollador Python}{\textsc{Qvantel}}{Madrid}{}{
\begin{itemize}
\item Desarrollador backend en Django con mariadb, docker-compose y RabbitMQ.
\item Desarrollo de soluciones a medida con Python, Bottle, jQuery y robot framework.
\item Implantación de TDD en el ciclo de desarrollo.
\item Mantenimiento de proyectos y core con Python y bash.
\end{itemize}
}
\cventry{2018--2019}{Desarrollador Python}{\textsc{SMACH Team}}{Palma de Mallorca - Parc Bit}{}{Desarrollo de aplicaciones y backend}
\cventry{2017--2018}{Ingeniero de software}{\textsc{SmartUIB}}{Universidad de las Islas Baleares}{}{Creación y desarrollo proyectos en el marco de \textit{Smart Cities}, eficiencia energética e IoT.
\begin{itemize}
\item Web Services SOAP con el protocolo OCPP.
\item Integraciones con sensores y otros servicios mediante Python para el sistema de Sentilo.
\item Elasticsearch y Kibana para visualización de datos.
\item Desarrollo de APIs y dashboards para el control de dispositivos IoT.
\end{itemize}
}

\cventry{2014--2016}{Desarrollador full stack}{\textsc{Civitfun}}{Palma de Mallorca - Parc Bit}{}{Desarrollo de la web \href{http://civitfun.com/en/}{civitfun.com} especialmente en tareas de backend, integraciones y diseño y mantenimiento de bases de datos.}

\cventry{2013--2014}{Desarrollador web}{\textsc{Mola}}{Palma de Mallorca - Parc Bit}{Becado por Talentum Startups}{Me inicié en el desarrollo web gracias a las tareas de mantenimiento en backend y de diseño en los proyectos que se crearon en esta incubadora.}


\section{Méritos}
\cventry{2013}{Beca Talentum Startups}{\textsc{Telefónica}}{}{}{Talentum Startups es una beca para estudiantes que quieran mejorar la sociedad a través de la tecnología. Los estudiantes seleccionados pueden desarrollar sus ideas en una Startup asociada al programa.}

%------------------------------------------------
%	SKILLZ
%----------------------------------------------------------------------------------------

\section{Habilidades}
\cvitem{Programación}{Python, JavaScript, Bash, PHP, Java}
\cvitem{Bases de Datos}{mariadb, MySQL, sqlite, MongoDB}
\cvitem{Frameworks}{Django, Flask, Bottle, jQuery, Laravel}
\cvitem{Frontend}{HTML/CSS, Bootstrap, QML}
\cvitem{}{Git, Linux, Docker, REST, JSON, RabbitMQ, XML, SOAP, MQTT, Kafka}

%----------------------------------------------------------------------------------------
%	LANGUAGES SECTION
%----------------------------------------------------------------------------------------

\section{Idiomas}

\cvitem{Español}{Lengua materna}
\cvitem{Catalán}{Nivel B2}
\cvitem{Inglés}{Nivel B2}

%----------------------------------------------------------------------------------------
%	INTERESTS SECTION
%----------------------------------------------------------------------------------------

\section{Datos de interés}
\cvitem{}{Disponibilidad para viajar al extranjero}
\cvitem{}{Permiso de conducir tipo B}



%----------------------------------------------------------------------------------------

\end{document}
